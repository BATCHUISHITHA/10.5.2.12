\let\negmedspace\undefined
\let\negthickspace\undefined
\documentclass[journal,12pt,onecolumn]{IEEEtran}
\usepackage{cite}
\usepackage{amsmath,amssymb,amsfonts,amsthm}
\usepackage{algorithmic}
\usepackage{graphicx}
\usepackage{textcomp}
\usepackage{xcolor}
\usepackage{txfonts}
\usepackage{listings}
\usepackage{enumitem}
\usepackage{mathtools}
\usepackage{gensymb}
\usepackage[breaklinks=true]{hyperref}
\usepackage{tkz-euclide} % loads  TikZ and tkz-base
\usepackage{listings}



\newtheorem{theorem}{Theorem}[section]
\newtheorem{problem}{Problem}
\newtheorem{proposition}{Proposition}[section]
\newtheorem{lemma}{Lemma}[section]
\newtheorem{corollary}[theorem]{Corollary}
\newtheorem{example}{Example}[section]
\newtheorem{definition}[problem]{Definition}
%\newtheorem{thm}{Theorem}[section] 
%\newtheorem{defn}[thm]{Definition}
%\newtheorem{algorithm}{Algorithm}[section]
%\newtheorem{cor}{Corollary}
\newcommand{\BEQA}{\begin{eqnarray}}
\newcommand{\EEQA}{\end{eqnarray}}
\newcommand{\define}{\stackrel{\triangle}{=}}
\theoremstyle{remark}
\newtheorem{rem}{Remark}
%\bibliographystyle{ieeetr}
\begin{document}
%
\providecommand{\pr}[1]{\ensuremath{\Pr\left(#1\right)}}
\providecommand{\prt}[2]{\ensuremath{p_{#1}^{\left(#2\right)} }}        % own macro for this question
\providecommand{\qfunc}[1]{\ensuremath{Q\left(#1\right)}}
\providecommand{\sbrak}[1]{\ensuremath{{}\left[#1\right]}}
\providecommand{\lsbrak}[1]{\ensuremath{{}\left[#1\right.}}
\providecommand{\rsbrak}[1]{\ensuremath{{}\left.#1\right]}}
\providecommand{\brak}[1]{\ensuremath{\left(#1\right)}}
\providecommand{\lbrak}[1]{\ensuremath{\left(#1\right.}}
\providecommand{\rbrak}[1]{\ensuremath{\left.#1\right)}}
\providecommand{\cbrak}[1]{\ensuremath{\left\{#1\right\}}}
\providecommand{\lcbrak}[1]{\ensuremath{\left\{#1\right.}}
\providecommand{\rcbrak}[1]{\ensuremath{\left.#1\right\}}}
\newcommand{\sgn}{\mathop{\mathrm{sgn}}}
\providecommand{\abs}[1]{\left\vert#1\right\vert}
\providecommand{\res}[1]{\Res\displaylimits_{#1}} 
\providecommand{\norm}[1]{\left\lVert#1\right\rVert}
%\providecommand{\norm}[1]{\lVert#1\rVert}
\providecommand{\mtx}[1]{\mathbf{#1}}
\providecommand{\mean}[1]{E\left[ #1 \right]}
\providecommand{\cond}[2]{#1\middle|#2}
\providecommand{\fourier}{\overset{\mathcal{F}}{ \rightleftharpoons}}
\newenvironment{amatrix}[1]{%
  \left(\begin{array}{@{}*{#1}{c}|c@{}}
}{%
  \end{array}\right)
}
%\providecommand{\hilbert}{\overset{\mathcal{H}}{ \rightleftharpoons}}
%\providecommand{\system}{\overset{\mathcal{H}}{ \longleftrightarrow}}
	%\newcommand{\solution}[2]{\textbf{Solution:}{#1}}
\newcommand{\solution}{\noindent \textbf{Solution: }}
\newcommand{\cosec}{\,\text{cosec}\,}
\providecommand{\dec}[2]{\ensuremath{\overset{#1}{\underset{#2}{\gtrless}}}}
\newcommand{\myvec}[1]{\ensuremath{\begin{pmatrix}#1\end{pmatrix}}}
\newcommand{\mydet}[1]{\ensuremath{\begin{vmatrix}#1\end{vmatrix}}}
\newcommand{\myaugvec}[2]{\ensuremath{\begin{amatrix}{#1}#2\end{amatrix}}}
\providecommand{\rank}{\text{rank}}
\providecommand{\pr}[1]{\ensuremath{\Pr\left(#1\right)}}
\providecommand{\qfunc}[1]{\ensuremath{Q\left(#1\right)}}
	\newcommand*{\permcomb}[4][0mu]{{{}^{#3}\mkern#1#2_{#4}}}
\newcommand*{\perm}[1][-3mu]{\permcomb[#1]{P}}
\newcommand*{\comb}[1][-1mu]{\permcomb[#1]{C}}
\providecommand{\qfunc}[1]{\ensuremath{Q\left(#1\right)}}
\providecommand{\gauss}[2]{\mathcal{N}\ensuremath{\left(#1,#2\right)}}
\providecommand{\diff}[2]{\ensuremath{\frac{d{#1}}{d{#2}}}}
\providecommand{\myceil}[1]{\left \lceil #1 \right \rceil }
\newcommand\figref{Fig.~\ref}
\newcommand\tabref{Table~\ref}
\newcommand{\sinc}{\,\text{sinc}\,}
\newcommand{\rect}{\,\text{rect}\,}
%%
%	%\newcommand{\solution}[2]{\textbf{Solution:}{#1}}
%\newcommand{\solution}{\noindent \textbf{Solution: }}
%\newcommand{\cosec}{\,\text{cosec}\,}
%\numberwithin{equation}{section}
%\numberwithin{equation}{subsection}
%\numberwithin{problem}{section}
%\numberwithin{definition}{section}
%\makeatletter
%\@addtoreset{figure}{problem}
%\makeatother

%\let\StandardTheFigure\thefigure
\let\vec\mathbf

\bibliographystyle{IEEEtran}





\bigskip

\renewcommand{\thefigure}{\theenumi}
\renewcommand{\thetable}{\theenumi}
%\renewcommand{\theequation}{\theenumi}

Q:Two APs have the same common difference.The difference between their $100${th} terms is 100,what is the difference between their $1000${th} terms?

\solution

Let us assume given two APs(Arithmetic Progressions) as X and Y whose first terms are given by $x(1)$ and $y(1)$. Let $d$ be the common difference of the APs. 
\\
We know that, $n$th term of an AP is given by $x(n) = x(1)+(n-1)d$.
\\
Given that the difference between the $100$th terms is $100$.
\begin{align}
x(100) - y(100) &= 100 \\
(x(1) + 99d) - (y(1) + 99d) &= 100
(\text{since $999d$ cancels out, the equation will be}) \\
x(1) - y(1) &= 100
\end{align}
Now to find the difference between the $1000$th terms of APs;
\begin{align}
x(1000) - y(1000) &= (x(1) + 999d) - (y(1) + 999d)
(\text{since $999d$ cancels out, the equation will be})\\
&= x(1) - y(1)\\
&= 100  
(\text{from the above equation})
\end{align}
Therefore,the difference between the 1000th terms of two given APs is 100.\\

We know that,

The Z-transform of a discrete signal $x(n)$ is given by:
\[ X(z) = \mathcal{Z}\{x(n)\} = \sum_{n=-\infty}^{\infty} x(n)z^{-n} \]
\\ Considering x(n) and y(n) as $n^{th}$ terms of the APs(Arithmetic Progressions), Z-transform for x(n) and y(n) can be given by
\begin{align}
X(z) = \mathcal{Z}\{x(n)\} 
&= \sum_{n=-\infty}^{\infty} x(n)z^{-n} \\
&=\sum_{n=1}^{n}x(n)z^{-n} \\
&=x(1)z^{-1}+x(2)z^{-2}+ \ldots +x(n)z^{-n} \\
&=x(1)z^{-1}+(x(1)+d)z^{-2}+ \ldots +(x(1)+(n-1)d)z^{-n} \\
&=x(1)[z^{-1} + z^{-2} + \ldots + z^{-n}] + d[1.z^{-2}+\ldots+(n-1).z^{-n}] \\
&=x(1)\cdot U(z) + d[-z \cdot \frac{d(U(z))}{dz} - U(z)]\\
\end{align}
$X(z) = [x(1)-d]\cdot U(z) + d[-z \cdot \frac {d(U(z))}{dz}]  $ \\

similarly \\

$Y(z) = [y(1)-d]\cdot U(z) + d[-z \cdot \frac {d(U(z))}{dz}] $ \\

\begin{table}
    \centering
    \begin{tabular}{|c|c|c|c|}
    \hline
       \textbf{Variable}& \textbf{Description}& \textbf{Value}\\\hline
         $X,Y$& Two given APs &none\\\hline
          $x(n)$,$y(n)$& $n^{th}$ term of X, $n^{th}$ term of Y&none\\\hline
         $n$&   position of the term in the AP&none\\\hline
          $d$&common difference between the terms of AP&none\\\hline
          $X(z)$& z-transform of x(n)&$[x(1)-d]\cdot U(z) + d[-z \cdot \frac {d(U(z))}{dz}]$ \\\hline
         $Y(z)$& z-transform of y(n)&$[y(1)-d]\cdot U(z) + d[-z \cdot \frac {d(U(z))}{dz}]$\\\hline
          $U(z)$& z-transform of u(n)& $\sum_{ n=1}^{\infty} z^{-n}$\\\hline
         $\frac{d(U(z))}{dz}$& Derivative of U(z)&$-\sum_{n=1}^{\infty}nz^{-n-1}$\\\hline
         
    \end{tabular}
    \caption{\textbf{VARIABLES AND THEIR VALUES}}
    \label{tab:my_label}
\end{table}

\end{document}
